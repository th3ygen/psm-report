\documentclass[conference]{IEEEtran}
\IEEEoverridecommandlockouts
% The preceding line is only needed to identify funding in the first footnote. If that is unneeded, please comment it out.
\usepackage{cite}
\usepackage{amsmath,amssymb,amsfonts}
\usepackage{algorithmic}
\usepackage{graphicx}
\usepackage{textcomp}
\usepackage{xcolor}
\usepackage{url}
\def\BibTeX{{\rm B\kern-.05em{\sc i\kern-.025em b}\kern-.08em
    T\kern-.1667em\lower.7ex\hbox{E}\kern-.125emX}}
\begin{document}

\title{Web-based IoT Data Dashboard and Analytics\\}

\author{\IEEEauthorblockN{MUHAMMAD AIDIL SYAZWAN BIN HAMDAN}
(Student)\\
\IEEEauthorblockA{\textit{FACULTY OF COMPUTING} \\
UNIVERSITY MALAYSIA PAHANG\\
26600, PEKAN, KUANTAN, PAHANG \\
Email: th3ygen@gmail.com}
\and
\IEEEauthorblockN{Dr. SYAFIQ FAUZI BIN KAMARULZAMAN}
(Supervisor)\\
\IEEEauthorblockA{\textit{FACULTY OF COMPUTING} \\
UNIVERSITY MALAYSIA PAHANG\\
26600, PEKAN, KUANTAN, PAHANG \\
Email: syafiq29@ump.edu.my}
}

\maketitle

\begin{IEEEkeywords}
iot, machine-to-machine, m2m, big data, data analytics, data dashboard, data visualization
\end{IEEEkeywords}

\section{Introduction}
Internet of Things

\subsection{Background}
The Internet of Things (IoT) is the network or medium for physical devices
to communicate and exchange data over the Internet without the need of
human-to-machine interaction \cite{b1}. The concept of IoT has been around
for a long time and it has contributed a significant impact on the
development of world technology. Nowadays, almost everything around
us is an IoT device such as smartwatch, home security system,
and even the famous voice assistant, Alexa, is an IoT device.
In short, any physical device that is accessible thru the Internet,
exchanges information between other devices, and does not require human
intervention is considered an IoT device. According to the statistics,
throughout the year since the first of the IoT era, total things connected
are increasing drastically. In 2016, total things connected are roughly
around 4.7 billion, then in 2021, the number reached a total of 11.6 billion
of IoT devices \cite{b2}.

\subsection{Problem statements}

\subsection{Objectives}


\begin{thebibliography}{00}
\bibitem{b1} "What is IoT? Defining the Internet of Things (IoT)" [Online] Available: https://www.aeris.com/in/what-is-iot/

[Accessed: 16 April 2021]
\bibitem{b2} "Prediction 2021: Technology Trends that will drive IoT growth", February 6, 2021, [Online] Available: \url{https://www.europeanbusinessreview.com/prediction-2021-technology-trends-that-will-drive-iot-growth/#:~:text=From%20mere%204.7%20billion%20things,multi%2Dfolds%20in%20coming%20years.}

[Accessed: 16 April 2021]

\end{thebibliography}
\vspace{12pt}

\end{document}